\section{The Monotonic Ideal Gas}%
\label{sec:MIG}
This chapter will concern gases that are sufficiently dilute as to have
negligible interactions with other particles. In addition, these molecules will
be monotonic i.e.\ only consist of one atom. Therefore, no rotational or
vibrational energy will be taken into consideration. Furthermore, we will assume
that the partition function of a molecule can be decoupled into its individual
energy modes.

Assumptions
\begin{itemize}
	\item Gas is dilute enough to neglect individual particle interactions.
	\item Individual energy modes in the atom are decoupled.
\end{itemize}
\section{The Individual Molecular Partition Functions}%
\label{sec:IMPF}
First we must assess the different modes of energy that a molecule can carry.
Since there are no interactions, there is no potential energy in the system
unless a body force were to be acting on the entire system. The only available
modes for storing energy into the system are then,
\begin{enumerate}
	\item Translational
	\item Electronic
	\item Nuclear.
\end{enumerate}
The partition functions are then,
\begin{align*}
	q_{mol} &= q_{trans}q_{elec}q_{nucl}\\
	\mathcal{Q} &= \frac{q_{mol}^N}{N!}\\
	\mathcal{Q} &= \frac{(q_{trans}q_{elec}q_{nucl})^N}{N!}.
\end{align*}

\subsection{Translational Partition Functions}
We begin with the exploration of the translational partition function. From one
of the first problems solved in quantum mechanics, the particle in an infinite
potential box, we know,
\begin{equation*}
	\epsilon_{n_x,n_y,n_z} = \frac{h^{2}}{8ma^{2}}(n_x^2 + n_y^2 + n_z^2).
\end{equation*}
From here we can directly plug in $\epsilon$ into $q_{trans}$.
\begin{align*}
	q_{trans} &= \sum_{j,k,l}{e^{-\beta (\epsilon_{n_x} + \epsilon_{n_y} +
	\epsilon_{n_z})}}\\
			  &= \sum_i{e^{-\beta \epsilon_{n_i}}} \sum_j{e^{-\beta
			  \epsilon_{n_j}}} \sum_l{e^{-\beta \epsilon_{n_l}}}\\
			  &= \left( \sum_{i=0}^{\infty}{e^{-\beta
			  \epsilon_{n_l}}}\right)^3\\
			  &= \left( \sum_{i=0}^{\infty}{\text{exp}\left(-\frac{\beta h^2
			  n^{2}}{8 m a^{2}}\right)}\right)^3
\end{align*}
\emph{This summation cannot be expressed exactly analytically.} However, the
density of available states is so large, that the available energy level can be
treated as continuous which means the summation can be approximated by an
integral. The difference between two energy levels is given by,
\begin{equation*}
	\Delta = \frac{\beta h^2 (n_x + 1)^2}{8ma^{2}} - \frac{\beta h^2
	n_x^{2}}{8ma^{2}} = \frac{\beta h^2 (2n_x + 1)}{8ma^{2}}.
\end{equation*}
At $T=300K$, $m=10^{-22}$, and $a=10cm$, $\Delta$ is on the order of $(2n_x + 1)
\cdot 10^{-20}$. For these conditions $n_x \approx 10^{10}$. As one can see then
the approximation is a very good one.

Thus, $q_{trans}$ becomes,
\begin{equation*}
	\left(\int_0^{\infty}{e^{-\beta h^2 n^2 / 8ma^{2}}\text{d}n}\right)^3 =
	\left(\frac{2\pi mkT}{h^{2}}\right)^{3/2} V.
\end{equation*}
$V$ represents the volume or $a^{3}$. This result follows that the integrand and
lower limit of the integral forms the error function. Another way to come to
this result is to use the density of states approximation where the density of
transitional energy levels is,
\begin{equation*}
	\omega(\epsilon)\text{d}\epsilon = \frac{\pi}{4}
	\left(\frac{8ma^{2}}{h^{2}}\right)^{3/2} \epsilon^{1/2}\text{d}\epsilon.
\end{equation*}
Some observations are worth noting. First that the average momentum $p$ is
$(mkT)^{1/2}$. This comes from equating the translational energy stated above
using an ensemble average and the momentum form of translational energy. From
here, we can see that $q_{trans}$ is,
\begin{equation*}
	\frac{V}{(h/p)^3} = \frac{V}{\Lambda^{3}}.
\end{equation*}
$\Lambda$ is simple the DeBroglie wavelength.

\subsection{Electronic Partition Functions}
For the electronic energy we use energy levels so that,
\begin{equation*}
	q_{elec} = \sum_i{\omega_i e^{-\beta \epsilon_i}}.
\end{equation*}
We then fix the ground state energy to zero which gives us,
\begin{equation*}
	q_{elec} = \omega_1 + \sum_{i=2}{\omega_i e^{-\beta \Delta\epsilon_{1i}}}.
\end{equation*}
The argument in the exponential is usually large since the energy difference is
on the order of electron volts. Specific cases like the halogen series require
more than just the first term, however. Regardless, the sum converges quickly
and energy levels of available states for many atoms and materials have been
tabulated. The following is usually enough,
\begin{equation*}
	q_{elec} = \omega_1 + \omega_2 e^{-\beta \Delta\epsilon_{12}}.
\end{equation*}

\subsection{Nuclear Partition Functions}
While there are excited nuclear states to access them requires temperatures of
$10^{10}K$. Thus only the ground state degeneracy matters and
\begin{equation*}
	q_{nucl} = \omega_1.
\end{equation*}
For most cases the nuclear partition function is just omitted.

\section{Thermodynamic Relations with the Ideal Gas}%
\label{sec:RIG}
Taking the definition of the Helmholtz energy in terms of the canonical ensemble
we have for the monotonic ideal gas,
\begin{align*}
	A(N,V,T) &= -kT\ln{\mathcal{Q}}\\
			 \quad\\
			 &= -kT\ln{\left[ \left(\frac{V}{\Lambda^{3}}\right)^N(\omega_1 +
			 \omega_2e^{-\beta\Delta\epsilon_{12}})^N \frac{1}{N!} \right]}\\
			 \quad\\
			 &= -NkT\ln{\frac{V}{\Lambda^{3}}} -NkT\ln{(\omega_1 +
			 \omega_2e^{-\beta\Delta\epsilon_{12}})} +NkT\ln{N} - NkT\\
			 \quad\\
			 &= NkT\left( -\ln{\frac{V}{\Lambda^{3}}} -\ln{(\omega_1 +
			 \omega_2e^{-\beta\Delta\epsilon_{12}})} +\ln{N} - 1 \right).
\end{align*}
The electronic term is small so,
\begin{align*}
	A &\approx -NkT\left(\ln{\frac{V}{\Lambda^{3}}} -\ln{N} + 1 \right)\\
	\text{or}\\
	A &\approx -NkT\left(\ln{\frac{V}{N\Lambda^{3}}} +1 \right).
\end{align*}

At this point it is helpful to note that,
\begin{equation*}
	\ln{\mathcal{Q}} = N\left(\ln{\frac{V}{N\Lambda^{3}}}(\omega_1 +
		\omega_2e^{-\beta\Delta\epsilon_{12}}) +1 \right).
\end{equation*}

Pressure is given by,
\begin{equation*}
	p = kT\left(\frac{\partial \ln{\mathcal{Q}}}{\partial T}\right)_{N,V} =
	\frac{NkT}{V}.
\end{equation*}

Energy is,
\begin{equation*}
	E = kT^2 \left(\frac{\partial \ln{\mathcal{Q}}}{\partial T}\right)_{N,V} =
	\frac{3}{2}NkT + \frac{N\omega_2\Delta\epsilon_{12}
	e^{-\beta\Delta\epsilon_{12}}} {q_{elec}}
\end{equation*}

Entropy is,
\begin{align*}
	S &= \frac{E}{T} - \frac{A}{T}\\
	\quad\\
	S &= \frac{3}{2}Nk + \frac{N\omega_2\Delta\epsilon_{12}
	e^{-\beta\Delta\epsilon_{12}}}{q_{elec}T} - NkT\left(
	-\ln{\frac{V}{\Lambda^{3}}} -\ln{(\omega_1 +
	\omega_2e^{-\beta\Delta\epsilon_{12}})} +\ln{N} - 1 \right)\\
	\quad\\
	  &= Nk\left( \frac{5}{2} + \ln{\frac{V}{N\Lambda^{3}}}\right) + Nk\ln(\omega_1 +
		\omega_2e^{-\beta\Delta\epsilon_{12}}) + \frac{N\omega_2\Delta\epsilon_{12}
		e^{-\beta\Delta\epsilon_{12}}} {q_{elec}}\\
	\quad\\
	  &= Nk\left( \frac{5}{2} + \ln{\frac{V}{N\Lambda^{3}}}\right) + S_{elec}.
\end{align*}

Using,
\begin{equation*}
	\mu(T, p) = -kT \left(\frac{\partial \ln{\mathcal{Q}}}{\partial
	N}\right)_{V,T},
\end{equation*}
one can similarly find the chemical potential. The procedure is the same for
all thermodynamic properties.
