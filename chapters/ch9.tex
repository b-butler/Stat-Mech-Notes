Like other properties such as pressure, temperature, and free energy can be
calculated from the partition function of a system, chemical equilibrium can be
derived from them as well.

\section{Chemical Equilibruim in Terms of Partition Functions}%
\label{sec:chemeqetpf}
We first note that chemical equilibrium in a reaction or phase change occurs
when,
\begin{equation*}
	\mu_I = \mu_{II},
\end{equation*}
or when chemical potentials are equal between the products and reactants or
between the two phases. This holds for multiple reactions and multiphase
behavior as well. One can also derive the necessary relations from a free energy
argument. Take an arbitrary homogeneous gas reaction,
\begin{equation*}
	\nu_A A + \nu_B B \kdf \nu_C C + \nu_D D.
\end{equation*}
We assume that the reaction is at equilibrium in a rigid thermostatted
container. By definition of the canonical ensemble this means that the Helmholtz
free energy is at a minimum. By the conservation of mass and the definition of
equilibrium we know
\begin{equation*}
	\nu_C C + \nu_D D - \\nu_A A + \nu_B B = 0.
\end{equation*}
Taking the definition of the Helmholtz free energy as,
\begin{equation*}
	\d{A} = -S\d{T} - p\d{V} + \sum_{j}{\mu_j \d{N_{j}}},
\end{equation*}
and using the fact that we are at constant $T$ and $V$, we get
\begin{equation*}
	\d{A} = \sum_{j}{\mu_j \d{N_{j}}} = {\left(\sum_{j}{\nu_j
	\mu_{j}}\right)}\d{\lambda},
\end{equation*}
where $\nu_j \d{\lambda} = \d{N_{j}}$. Since at equilibrium we are at a minimum
with respect to $A$, we know that $\partial A/ \partial \lambda$ must be zero,
and
\begin{equation*}
	\sum_{j}{\nu_j \mu_j} = 0.
\end{equation*}
Now to put statistical mechanics back into it we need an expression for the
partition function for the multicomponent system.  if the systems is an ideal
gas, the partition function of the system is just the multiplication of the
\begin{align*}
	Q_{tot} &= Q_{N_{A}} Q_{N_B} Q_{N_C} Q_{N_{D}}\\
			&= \frac{q_{A}^{N_A}}{N_A !} \frac{q_{B}^{N_B}}{N_B !}
			   \frac{q_{C}^{N_C}}{N_C !} \frac{q_{D}^{N_D}}{N_D !}.
\end{align*}
Using the previously derived relation,
\begin{align*}
	\mu_A &= -kT {\left( \frac{\partial \ln{\Q}}{\partial N_{A}}\right}_{N_j, V,
			T} = -kT {\left( \ln{\left(\frac{q_{A}}{N_A !}\right)} /
			\partial N_A \right}\\
		  &= -kT(N_{A} \ln{q_{A}} - N_{A}\ln{N_{A}} + N_{A}) / \partial N_A
		  = -kT {\left( N_A \ln{\left[\frac{q_A}{N_{A}}\right]} + N_A \right)} /
		  \partial N_A \\
		  &= -kT {\left( \ln{\left[ \frac{q_A}{N_A} \right]} - 1 + 1 \right)}\\
		  &= -kT \ln{\left[ \frac{q_A}{N_A} \right]}.
\end{align*}
The second to last line uses the product rule.

